% Options for packages loaded elsewhere
\PassOptionsToPackage{unicode}{hyperref}
\PassOptionsToPackage{hyphens}{url}
\PassOptionsToPackage{dvipsnames,svgnames,x11names}{xcolor}
%
\documentclass[
  letterpaper,
  DIV=11,
  numbers=noendperiod]{scrreprt}

\usepackage{amsmath,amssymb}
\usepackage{iftex}
\ifPDFTeX
  \usepackage[T1]{fontenc}
  \usepackage[utf8]{inputenc}
  \usepackage{textcomp} % provide euro and other symbols
\else % if luatex or xetex
  \usepackage{unicode-math}
  \defaultfontfeatures{Scale=MatchLowercase}
  \defaultfontfeatures[\rmfamily]{Ligatures=TeX,Scale=1}
\fi
\usepackage{lmodern}
\ifPDFTeX\else  
    % xetex/luatex font selection
\fi
% Use upquote if available, for straight quotes in verbatim environments
\IfFileExists{upquote.sty}{\usepackage{upquote}}{}
\IfFileExists{microtype.sty}{% use microtype if available
  \usepackage[]{microtype}
  \UseMicrotypeSet[protrusion]{basicmath} % disable protrusion for tt fonts
}{}
\makeatletter
\@ifundefined{KOMAClassName}{% if non-KOMA class
  \IfFileExists{parskip.sty}{%
    \usepackage{parskip}
  }{% else
    \setlength{\parindent}{0pt}
    \setlength{\parskip}{6pt plus 2pt minus 1pt}}
}{% if KOMA class
  \KOMAoptions{parskip=half}}
\makeatother
\usepackage{xcolor}
\setlength{\emergencystretch}{3em} % prevent overfull lines
\setcounter{secnumdepth}{5}
% Make \paragraph and \subparagraph free-standing
\makeatletter
\ifx\paragraph\undefined\else
  \let\oldparagraph\paragraph
  \renewcommand{\paragraph}{
    \@ifstar
      \xxxParagraphStar
      \xxxParagraphNoStar
  }
  \newcommand{\xxxParagraphStar}[1]{\oldparagraph*{#1}\mbox{}}
  \newcommand{\xxxParagraphNoStar}[1]{\oldparagraph{#1}\mbox{}}
\fi
\ifx\subparagraph\undefined\else
  \let\oldsubparagraph\subparagraph
  \renewcommand{\subparagraph}{
    \@ifstar
      \xxxSubParagraphStar
      \xxxSubParagraphNoStar
  }
  \newcommand{\xxxSubParagraphStar}[1]{\oldsubparagraph*{#1}\mbox{}}
  \newcommand{\xxxSubParagraphNoStar}[1]{\oldsubparagraph{#1}\mbox{}}
\fi
\makeatother


\providecommand{\tightlist}{%
  \setlength{\itemsep}{0pt}\setlength{\parskip}{0pt}}\usepackage{longtable,booktabs,array}
\usepackage{calc} % for calculating minipage widths
% Correct order of tables after \paragraph or \subparagraph
\usepackage{etoolbox}
\makeatletter
\patchcmd\longtable{\par}{\if@noskipsec\mbox{}\fi\par}{}{}
\makeatother
% Allow footnotes in longtable head/foot
\IfFileExists{footnotehyper.sty}{\usepackage{footnotehyper}}{\usepackage{footnote}}
\makesavenoteenv{longtable}
\usepackage{graphicx}
\makeatletter
\def\maxwidth{\ifdim\Gin@nat@width>\linewidth\linewidth\else\Gin@nat@width\fi}
\def\maxheight{\ifdim\Gin@nat@height>\textheight\textheight\else\Gin@nat@height\fi}
\makeatother
% Scale images if necessary, so that they will not overflow the page
% margins by default, and it is still possible to overwrite the defaults
% using explicit options in \includegraphics[width, height, ...]{}
\setkeys{Gin}{width=\maxwidth,height=\maxheight,keepaspectratio}
% Set default figure placement to htbp
\makeatletter
\def\fps@figure{htbp}
\makeatother

\KOMAoption{captions}{tableheading}
\makeatletter
\@ifpackageloaded{bookmark}{}{\usepackage{bookmark}}
\makeatother
\makeatletter
\@ifpackageloaded{caption}{}{\usepackage{caption}}
\AtBeginDocument{%
\ifdefined\contentsname
  \renewcommand*\contentsname{Table of contents}
\else
  \newcommand\contentsname{Table of contents}
\fi
\ifdefined\listfigurename
  \renewcommand*\listfigurename{List of Figures}
\else
  \newcommand\listfigurename{List of Figures}
\fi
\ifdefined\listtablename
  \renewcommand*\listtablename{List of Tables}
\else
  \newcommand\listtablename{List of Tables}
\fi
\ifdefined\figurename
  \renewcommand*\figurename{Figure}
\else
  \newcommand\figurename{Figure}
\fi
\ifdefined\tablename
  \renewcommand*\tablename{Table}
\else
  \newcommand\tablename{Table}
\fi
}
\@ifpackageloaded{float}{}{\usepackage{float}}
\floatstyle{ruled}
\@ifundefined{c@chapter}{\newfloat{codelisting}{h}{lop}}{\newfloat{codelisting}{h}{lop}[chapter]}
\floatname{codelisting}{Listing}
\newcommand*\listoflistings{\listof{codelisting}{List of Listings}}
\makeatother
\makeatletter
\makeatother
\makeatletter
\@ifpackageloaded{caption}{}{\usepackage{caption}}
\@ifpackageloaded{subcaption}{}{\usepackage{subcaption}}
\makeatother

\ifLuaTeX
  \usepackage{selnolig}  % disable illegal ligatures
\fi
\usepackage{bookmark}

\IfFileExists{xurl.sty}{\usepackage{xurl}}{} % add URL line breaks if available
\urlstyle{same} % disable monospaced font for URLs
\hypersetup{
  pdftitle={Proyectos de Bases de Datos},
  pdfauthor={Pedro Villa Casas},
  colorlinks=true,
  linkcolor={blue},
  filecolor={Maroon},
  citecolor={Blue},
  urlcolor={Blue},
  pdfcreator={LaTeX via pandoc}}


\title{Proyectos de Bases de Datos}
\author{Pedro Villa Casas}
\date{2015-07-10}

\begin{document}
\maketitle

\renewcommand*\contentsname{Table of contents}
{
\hypersetup{linkcolor=}
\setcounter{tocdepth}{2}
\tableofcontents
}

\bookmarksetup{startatroot}

\chapter{Preface}\label{preface}

This is a Quarto book.

To learn more about Quarto books visit
\url{https://quarto.org/docs/books}.

\bookmarksetup{startatroot}

\chapter{Agroquímicos ``Gallo Giro''}\label{agroquuxedmicos-gallo-giro}

\section{Introducción}\label{introducciuxf3n}

La empresa de agroquímicos ``Gallo Giro'' necesita controlar el
movimiento de sus productos en toda la región noroeste del país. Para
esto está organizada en diferentes zonas, que agrupan varios municipios.
Es necesario llevar un registro de los TI, las ventas y las compras de
productos, así como el control de los clientes y proveedores.

\section{Productos}\label{productos}

Los productos que maneja la empresa pertenecen a varias categorías,
tales como:

\begin{itemize}
\tightlist
\item
  \textbf{Fertilizantes}
\item
  \textbf{Semillas}
\item
  \textbf{Herbicidas}
\item
  \textbf{Fungicidas}
\item
  \textbf{Insecticidas}
\item
  \textbf{Otros agroquímicos}
\end{itemize}

\subsection{Presentaciones de
Productos}\label{presentaciones-de-productos}

Cada producto puede venir en diferentes tamaños o presentaciones, por
ejemplo:

\begin{itemize}
\tightlist
\item
  \textbf{Faena 1lt}
\item
  \textbf{Faena 1/2 lt}
\item
  \textbf{Faena galón}
\item
  \textbf{Otros tamaños específicos}
\end{itemize}

Cada presentación es de hecho, el artículo que se vende como tal y tiene
un Código Universal de Producto (UPC) único. Puedes encontrar más
información sobre el UPC en
\href{https://es.wikipedia.org/wiki/C\%C3\%B3digo_Universal_de_Producto}{Wikipedia}.

\subsection{Componentes Activos}\label{componentes-activos}

Diferentes productos comerciales pueden tener el mismo componente
activo. Es importante llevar un registro de estos componentes para
asegurar un manejo adecuado y evitar duplicidades.

\subsection{Proveedores y Precios}\label{proveedores-y-precios}

Cada producto es provisto por un proveedor específico con un costo
determinado. La empresa vende estos productos a un precio diferente al
público. Es crucial mantener un registro actualizado de los costos y
precios de venta para asegurar la rentabilidad.

\subsection{Punto de Reorden}\label{punto-de-reorden}

Los productos tienen un punto de reorden, que es la cantidad mínima de
existencia antes de solicitar más producto. Este punto de reorden se
establece para cada sucursal y CEDIS, asegurando que siempre haya
suficiente inventario disponible para satisfacer la demanda.

\section{Sucursales y CEDIS}\label{sucursales-y-cedis}

La empresa tiene varias sucursales en el estado, que mantienen una
cierta cantidad de algunos productos en sus instalaciones. Otros
productos son mantenidos en los Centros de distribución (CEDIS) por su
volumen o peligrosidad. Un CEDIS mantiene una gran cantidad de productos
que distribuye sólo a las sucursales que de él dependen. Puede haber
varios CEDIS en una zona.

\section{Traslado de Inventario (TI)}\label{traslado-de-inventario-ti}

En ocasiones es necesario trasladar mercancía de un CEDIS a otro o de
una sucursal a su CEDIS, esto se conoce como Traslado de Inventario
(TI). Estos no pueden realizarse de sucursal a sucursal, ni a, o desde
un CEDIS ajeno. Un CEDIS nunca vende, una sucursal nunca compra. De cada
TI se necesita conocer el nombre del almacenista que solicita los
productos y de quien autoriza, fecha y hora de salida y de entrada.

\section{Procedimiento de Venta y
Entrega}\label{procedimiento-de-venta-y-entrega}

Cuando se realiza una venta se registra al vendedor, al almacenista que
da salida al producto y si fue a domicilio al repartidor. Se genera el
comprobante de venta que especifica una de 3 formas de entregar el
producto al cliente:

\begin{itemize}
\tightlist
\item
  \textbf{En Sucursal.} Si el producto y cantidad se tiene en sucursal.
  La venta se marca como entregada.
\item
  \textbf{En el CEDIS.} Si el producto o cantidad no se tiene en
  sucursal, o solo se maneja en CEDIS y el cliente quiere recogerlo en
  el CEDIS. En el comprobante de venta se imprime la dirección del CEDIS
  a donde debe acudir el cliente. Al entregarle al cliente la venta se
  marca como entregada.
\item
  \textbf{A domicilio}. El producto se entrega en el domicilio o campo
  donde se utilizará el producto. Se necesita registrar la dirección de
  entrega. Al regresar el repartidor de entregarle al cliente, la orden
  se marca como entregada.
\end{itemize}

\subsection{Procedimiento de Envío de
Productos}\label{procedimiento-de-envuxedo-de-productos}

Cuando un cliente compra algunos productos que quiere que los envíen a
domicilio y algunos se tienen en la sucursal por no ser peligrosos, pero
también compró productos que pueden ser tóxicos y por lo tanto se
mantienen en el CEDIS, lo más conveniente es que todos los productos
salgan del CEDIS. Esto se debe a: - \textbf{Consolidación del Envío:} Al
enviar todos los productos desde el CEDIS, se puede consolidar el envío
en una sola entrega, lo que reduce costos y simplifica la logística. -
\textbf{Seguridad:} Los productos tóxicos ya están en el CEDIS, que está
preparado para manejar y almacenar este tipo de productos de manera
segura. - \textbf{Eficiencia:} Evita la necesidad de coordinar múltiples
puntos de salida, lo que puede complicar la logística y aumentar el
riesgo de errores.

\section{Clientes}\label{clientes}

Los clientes pueden ser:

\begin{itemize}
\tightlist
\item
  \textbf{Doméstico:} personas que compran productos para su jardín o
  huerto familiar.
\item
  \textbf{Agricultor:} el que siembra grandes cantidades de tierra. A su
  vez, puede ser horticultor, fruticultor, productor de granos y
  semillas, etc.
\end{itemize}

\subsection{Crédito a Clientes}\label{cruxe9dito-a-clientes}

Los clientes pueden tener crédito en la empresa. Los días 10 del mes se
hace corte, que se utiliza para calcular los intereses sobre las
compras. El día 30 es el límite para pagar sin generar intereses. Todas
las compras son a 6 meses. Si un pago se retrasa, genera un cargo del
10\%.

\section{Puestos en la Empresa}\label{puestos-en-la-empresa}

Además del puesto de almacenista, la empresa maneja otros puestos clave,
tales como:

\begin{itemize}
\tightlist
\item
  \textbf{Gerente de Sucursal:} Responsable de la operación diaria de
  una sucursal, incluyendo la gestión de inventarios, atención al
  cliente y supervisión del personal.
\item
  \textbf{Gerente de CEDIS:} Encargado de la administración y operación
  del Centro de Distribución, asegurando el correcto almacenamiento y
  distribución de productos.
\item
  \textbf{Vendedor:} Encargado de atender a los clientes, procesar
  ventas y asesorar sobre los productos.
\item
  \textbf{Comprador:} Responsable de gestionar las compras de productos
  a proveedores, asegurando el abastecimiento adecuado.
\item
  \textbf{Repartidor:} Encargado de entregar los productos a domicilio o
  en el campo donde se utilizarán.
\item
  \textbf{Contador:} Maneja las finanzas de la empresa, incluyendo la
  facturación, pagos y gestión de créditos de los clientes.
\item
  \textbf{Supervisor de Zona:} Coordina las actividades de varias
  sucursales y CEDIS dentro de una zona específica.
\item
  \textbf{Personal de Atención al Cliente:} Encargado de resolver dudas
  y problemas de los clientes, así como de gestionar sus cuentas y
  créditos.
\end{itemize}

\section{Ejemplos}\label{ejemplos}

\begin{itemize}
\tightlist
\item
  El CEDIS ``Aeropuerto'' compra, a ``Fertilizantes del Bajío'',

  \begin{itemize}
  \tightlist
  \item
    Fertilizante triple 17 ``Ever Green'' en diferentes presentaciones:
    20 costales de 5 kilos, 100 bolsas de 1 kilo, 50 bolsas de medio
    kilo;
  \item
    20 cajas con 100 paquetes de 50 bolsitas de 10 gramos de
    fertilizante de liberación lenta ``Ever Grow'';
  \item
    Fertilizante foliar ``Beauty flower'': 100 botellas de 1 litro, 50
    botellas de un galón.
  \end{itemize}
\item
  Juan Méndez del CEDIS ``Imala'' solicita al CEDIS ``Aeropuerto'' un TI
  de:

  \begin{itemize}
  \tightlist
  \item
    40 sacos de 50 kilos de fertilizante Urea ``Tepeyac''.
  \item
    20 sacos de 50Kg de semilla de trigo
  \end{itemize}

  Felipe Mejía de ``Aeropuerto'' acepta y lo envía con Alberto Zamudio,
  en ``Imala'' recibe Alfonzo Cuadras.
\item
  La Ing. Brenda Solares compra a la sucursal ``Caballito'', via
  telefónica:

  \begin{itemize}
  \tightlist
  \item
    10 toneladas de NAC27 de ``Fertiberia''
  \end{itemize}

  Pide que se lo lleven al ``Campo Victoria''; lo cargan a su cuenta. El
  primer pago se retrasa por 3 meses. Por lo que se generan intereses de
  esos primeros meses. El resto de los pagos los realiza a tiempo.
\item
  El señor Manuel Castro compra para su huerto familiar:

  \begin{itemize}
  \tightlist
  \item
    3 bolsitas de 7 gramos de ácido giberélico ``Raligeb''\\
  \item
    1 botella de 1 litro de foliar ``EcoJal''\\
    Paga en efectivo.
  \end{itemize}
\end{itemize}

\section{Reportes}\label{reportes}

\subsection{Ventas}\label{ventas}

\begin{itemize}
\tightlist
\item
  Reporte de Ventas Diarias

  \begin{itemize}
  \tightlist
  \item
    Descripción: Muestra las ventas realizadas cada día, incluyendo
    detalles de los productos vendidos.
  \item
    Campos: Fecha, Producto, Variante, Cantidad Vendida, Ingresos
    Totales.
  \end{itemize}
\item
  Reporte de Ventas por Sucursal

  \begin{itemize}
  \tightlist
  \item
    Descripción: Muestra las ventas realizadas en cada sucursal,
    desglosadas por producto y variante.
  \item
    Campos: Sucursal, Producto, Variante, Cantidad Vendida, Ingresos
    Totales.
  \end{itemize}
\item
  Reporte de Ventas por Vendedor

  \begin{itemize}
  \tightlist
  \item
    Descripción: Muestra las ventas realizadas por cada vendedor,
    incluyendo detalles de los productos vendidos.
  \item
    Campos: Vendedor, Producto, Variante, Cantidad Vendida, Ingresos
    Totales.
  \end{itemize}
\item
  Reporte de Ventas por Cliente
\end{itemize}

Descripción: Muestra las ventas realizadas a cada cliente, incluyendo
detalles de los productos comprados. Campos: Cliente, Fecha de Venta,
Producto, Variante, Cantidad, Precio Total. Reporte de Ventas por
Categoría de Producto

Descripción: Muestra las ventas agrupadas por categoría de producto,
desglosadas por variante. Campos: Categoría, Producto, Variante,
Cantidad Vendida, Ingresos Totales. Reporte de Ventas Mensuales

Descripción: Muestra las ventas realizadas cada mes, incluyendo detalles
de los productos vendidos. Campos: Mes, Producto, Variante, Cantidad
Vendida, Ingresos Totales. Reporte de Ventas con Descuentos Aplicados

Descripción: Muestra las ventas en las que se aplicaron descuentos,
incluyendo detalles de los productos vendidos y el monto del descuento.
Campos: Fecha, Producto, Variante, Cantidad Vendida, Descuento Aplicado,
Ingresos Totales. Reporte de Ventas por Método de Pago

Descripción: Muestra las ventas desglosadas por método de pago
(efectivo, tarjeta, crédito, etc.). Campos: Método de Pago, Producto,
Variante, Cantidad Vendida, Ingresos Totales. Reporte de Ventas por
Región

Descripción: Muestra las ventas realizadas en diferentes regiones,
desglosadas por producto y variante. Campos: Región, Producto, Variante,
Cantidad Vendida, Ingresos Totales. Reporte de Ventas de Productos
Tóxicos

Descripción: Muestra las ventas de productos clasificados como tóxicos,
incluyendo detalles de los productos vendidos. Campos: Fecha, Producto,
Variante, Cantidad Vendida, Ingresos Totales.

Reporte de Rentabilidad por Producto

Descripción: Analiza la rentabilidad de cada producto considerando los
costos de adquisición, almacenamiento y distribución, comparados con los
ingresos generados por las ventas. Campos: Producto, Variante, Ingresos
Totales, Costos Totales, Margen de Rentabilidad. Reporte de Tendencias
de Ventas

Descripción: Identifica las tendencias de ventas a lo largo del tiempo,
mostrando patrones estacionales, picos de demanda y productos
emergentes. Campos: Producto, Variante, Ventas Mensuales, Tendencia
(Creciente/Decreciente). Reporte de Análisis de Clientes

Descripción: Segmenta a los clientes según su comportamiento de compra,
identificando los clientes más valiosos, los que compran con mayor
frecuencia y aquellos que han disminuido sus compras. Campos: Cliente,
Frecuencia de Compra, Valor Total de Compras, Segmento de Cliente (VIP,
Frecuente, Inactivo). Reporte de Eficiencia de Sucursales

Descripción: Evalúa la eficiencia de cada sucursal en términos de
ventas, costos operativos y satisfacción del cliente. Campos: Sucursal,
Ventas Totales, Costos Operativos, Índice de Satisfacción del Cliente,
Eficiencia Operativa. Reporte de Impacto de Promociones

Descripción: Analiza el impacto de las promociones y descuentos en las
ventas, identificando cuáles promociones fueron más efectivas y su
efecto en el margen de rentabilidad. Campos: Promoción, Producto,
Variante, Incremento en Ventas, Margen de Rentabilidad, ROI de la
Promoción. Reporte de Devoluciones y Reclamaciones

Descripción: Muestra las devoluciones y reclamaciones de productos,
identificando patrones y causas comunes para mejorar la calidad y el
servicio al cliente. Campos: Producto, Variante, Cantidad Devuelta,
Motivo de Devolución, Costo de Devoluciones. Reporte de Análisis de
Competencia

Descripción: Compara las ventas y precios de los productos con los de la
competencia, identificando oportunidades para ajustar estrategias de
precios y promociones. Campos: Producto, Variante, Precio Interno,
Precio de la Competencia, Diferencia de Precio, Ventas Comparativas.
Reporte de Proyección de Ventas

Descripción: Utiliza datos históricos y análisis predictivo para
proyectar las ventas futuras, ayudando en la planificación de
inventarios y estrategias de marketing. Campos: Producto, Variante,
Ventas Históricas, Proyección de Ventas, Factores de Influencia. Reporte
de Cumplimiento de Objetivos de Ventas

Descripción: Evalúa el desempeño de las ventas en relación con los
objetivos establecidos, identificando áreas de mejora y éxitos. Campos:
Producto, Variante, Objetivo de Ventas, Ventas Reales, Desviación del
Objetivo, Comentarios. Reporte de Análisis de Canales de Venta

Descripción: Analiza el desempeño de diferentes canales de venta (tienda
física, online, telefónica), identificando cuál canal es más efectivo
para cada tipo de producto. Campos: Canal de Venta, Producto, Variante,
Ventas Totales, Costos Asociados, Margen de Rentabilidad.

\bookmarksetup{startatroot}

\chapter{Pokémon Go}\label{pokuxe9mon-go}

Es un juego que usted quiere diseñar. La idea es que va recolectando
pokemones que se encuentra en el camino.

Cada pokémon puede ser de una especie, por ejemplo: Pika, Chorrito,
Rocacho, Dormilón, etc. Cada especie es de 1 o 2 tipos, por ejemplo:
Agua, fuego, planta, psiquico, etc.

Cada especie de pokemon tiene valores diferentes para algunas
características: Poder de Combate máximo, puntos de salud, poder de
defensa, poder de Ataque.

Cada especie puede aprender un conjunto de diferentes ataques de
diferente tipo: Trueno tipo eléctrico, Chorro tipo Agua, etc.

Los ataques pueden ser o rápidos o cargados. Los ataques rápidos se
pueden realizar rápidamente uno tras otro y cada vez van acumulando
energía para poder realizar el ataque cargado.

Cada bicho puede aprender un ataque rápido y un ataque cargado.

Cada especie es mas o menos resistente a ataques de cierto tipo de
pokémon: Los tipo Planta resisten los ataques tipo agua, pero no los
tipo Fuego.

Ejemplo:

\begin{longtable}[]{@{}lr@{}}
\toprule\noalign{}
\textbf{Anihilape} & \\
\midrule\noalign{}
\endhead
\bottomrule\noalign{}
\endlastfoot
\textbf{Debilidades} & \textbf{Factor} \\
Hada & 160\% \\
Volador & 160\% \\
Fantasma & 160\% \\
Psíquico & 160\% \\
\textbf{Resistencias} & \textbf{Factor} \\
Bicho & 39\% \\
Lucha & 39\% \\
Normal & 39\% \\
Veneno & 62\% \\
Roca & 62\% \\
\end{longtable}

Cada especie puede evolucionar en otra especie de bicho, teniendo un
árbol de evoluciones posibles. En la evolución algunos atributos pueden
variar, incluidos los ataques que sabe.

Los usuarios pueden recolectar varios bichos, incluso de la misma
especie, cada uno con sus propios valores de los diferentes atributos.
Los bichos pueden intercambiarse entre los usuarios. El usuario puede
deshacerse de cualquier bicho. El bicho puede ser transferido (se
pierde) o liberado en la ubicación actual del usuario(disponible durante
10 minutos para que otro usuario lo encuentre y trate de capturarlo). Un
bicho tiene un UUID con el cual se le puede dar seguimiento durante su
tiempo de vida.

Un usuario puede capturar cualquier bicho que aparezca en un radio de
50m.

El sistema genera los bichos en un radio de 50m de la ubicación de un
usuario. El \emph{spawn} está influenciado por: tiempo climático, si hay
nido, región, hora del dia, estado del clima. Un mismo bicho no puede
ser capturado por dos usuarios, aunque si pueden intentar hacerlo al
mismo tiempo (\emph{transacciones y bloqueo de recursos}). Un bicho
permanece en el mismo lugar de su \emph{spawn} por 10 minutos.

Por lo pronto, el sistema cambia el tiempo climático por zonas cada
hora.

Una región abarca una gran área geográfica: Norteamérica, Sudamérica,
Europa, Asia, Norte de África, etc.

Un usuario puede organizar combates contra otro para lo cual escoge a 3
de sus bichos. Los bichos se enfrentan uno a uno contra los del
contrincante realizando un serie de ataques entre si hasta que uno de
los dos queda sin puntos de salud. Luego pueden enfrentarse cualesquiera
otros dos bichos de los equipos en combate.

Cuando un pokémon es capturado proporciona cierta cantidad de caramelos
de la especie base de la cadena evolutiva.

Cada especie requiere cierta cantidad de caramelos para llegar a ella de
la especie anterior en la cadena evolutiva.

Los caramelos se utilizan para aumentar el Poder de Combate de un
pokémon. Se debe llevar el registro de cuantos caramelos de cada especie
tiene un entrenador.

El sistema debe llevar registro de la ubicación donde se recolectó cada
bicho y la propiedad de los usuarios, así como las estadísticas de
combate.

Amplíe este escenario para hacer mas completo el juego, de tal forma que
cambie la estructura de la base de datos: Pokeparadas, Gimnasios, Nidos,
etc. 

\section{Pendientes}\label{pendientes}

\begin{itemize}
\tightlist
\item
  Como se calcula el daño en combate
\end{itemize}

\bookmarksetup{startatroot}

\chapter{Comedor del Colegio Pink
Floyd}\label{comedor-del-colegio-pink-floyd}

El ``Colegio Pink Flor'' con kínder y primaria busca mejorar sus
servicios y lo han contratado a usted para desarrollar el sistema
computacional que controle el servicio de comedor que ofrece a sus
alumnos de lunes a viernes.

Inicialmente el sistema es independiente del resto de los que posee el
colegio, por lo tanto, necesita guardar información sobre los niños: su
nombre, edad, nivel y grado, alergias alimenticias; del padre o tutor:
su nombre, teléfono celular, teléfono y lugar de trabajo.

El nutriólogo ha seleccionado cuidadosamente cientos de alimentos para
el área de cocina. Para cada alimento se tiene una Receta que describe
el procedimiento e ingredientes para preparar una cantidad de porciones.

Cada semana el nutriólogo diseña un menú del cual los padres pueden
seleccionar lo que comerán sus hijos. El menú consiste de al menos 7
opciones de comida, 7 bebidas y 7 postres, para cada una de ellas el
nutriólogo calcula las calorías, carbohidratos, grasas y proteínas.
Sobre los ingredientes de los alimentos, es importante identificar
aquellos que son alergénicos. El nutriólogo puede reutilizar menúes.

Con al menos 3 días de anticipación, el padre puede seleccionar los
alimentos que recibirán sus hijos, que puede pagar en el momento o en
abonos. Es necesario pues llevar un control de pagos.

Los lunes y miércoles se hacen las compras de los ingredientes, es
importante generar la lista de compras, considerando lo que se ha
utilizado y aún queda en la cocina. De cada día de la semana se
contabiliza cuantos niños recibirán cada comida específica. De cada
comida se prepararán un múltiplo de las porciones que especifica la
receta. La cantidad de cada ingrediente se multiplica por este factor.

El niño come sólo lo que el padre apruebe del menú. Los niños pueden ser
alérgicos a ciertos ingredientes, por lo cual es importante conocer sus
alergias alimentarias.

\section{Ejemplos}\label{ejemplos-1}

\subsection{Ejemplo 1}\label{ejemplo-1}

Actualmente es la semana 20 y los padres pueden seleccionar de este
menú:

\begin{longtable}[]{@{}lll@{}}
\toprule\noalign{}
Comida & Bebida & Postre \\
\midrule\noalign{}
\endhead
\bottomrule\noalign{}
\endlastfoot
Mole poblano & Agua de melón & Gelatina sabores varios \\
Ceviche de soya & Agua de sandía & Pastel de 3 leches \\
Verduras al vapor con puré de papa & Agua de jamaica & Flan \\
Tacos de carne asada & Limonada & Fruta picada \\
Tamales de verdura & Agua de piña & Frutos rojos \\
Enchiladas verdes & Agua de pepino & Mangoneadas \\
Pollo a la plancha & Agua de horchata & Pastel de chocolate \\
\end{longtable}

\subsection{Ejemplo 2}\label{ejemplo-2}

El señor Ramírez tiene 2 hijos, no se quiere complicar y ordena lo mismo
para sus hijos; paga el total:

Órdenes de Juan para la semana 21 \textbar{} Día \textbar{} Comida
\textbar{} Bebida \textbar{} Postre \textbar{}
\textbar-------\textbar-------\textbar-------\textbar-------\textbar{}
\textbar{} \textbf{Lunes} \textbar{} Tacos de carne asada\textbar{} Agua
de horchata \textbar{} Frutos rojos\textbar{} \textbar{} \textbf{Martes}
\textbar{} Mole poblano \textbar{} Agua de pepino \textbar{}
Mangoneadas\textbar{} \textbar{} \textbf{Miércoles} \textbar{}
Enchiladas verdes\textbar{} Agua de jamaica \textbar{} Pastel de
chocolate\textbar{} \textbar{} \textbf{Jueves} \textbar{} Tamales de
verdura \textbar{} Agua de piña \textbar{} Frutos rojos\textbar{}
\textbar{} \textbf{Viernes} \textbar{} Ceviche de soya\textbar{} Agua de
melón\textbar{} Flan\textbar{}

\subsection{Ejemplo 3}\label{ejemplo-3}

El nutriólogo registra una nueva comida:

\textbf{Crêpes salados}

\textbf{Ingredientes} (Para 5 porciones)

\begin{itemize}
\tightlist
\item
  Harina de garbanzo: 110 g
\item
  Agua: 250 ml
\item
  Aceite de oliva virgen extra: 15 ml
\item
  Sal: 2 g
\item
  Cebollino: 10 gramos
\item
  Pimienta negra molida: 1 gramo
\item
  Aguacate: 1
\item
  Cebolla morada: 1
\item
  Pimiento rojo picante o dulce: 1
\end{itemize}

\textbf{Preparación:}

Disponer la harina de garbanzos en un cuenco o jarra con el agua\ldots{}

\ldots{} Continuar hasta terminar con toda la masa, rellenar con los
ingredientes preparados y servir.

\subsection{Ejemplo 4}\label{ejemplo-4}

Ana Flérida está en la fila, es su turno, pone su credencial de la
escuela en el sensor, en su pantalla y la de la barra aparece su orden
para ese día. Marko le sirve, le entrega y oprime el botón de entregado.
 

\bookmarksetup{startatroot}

\chapter{Club de Conquistadores}\label{club-de-conquistadores}

El ``Club de Conquistadores'' es una organización global que adiestra a
niños entre 10 y 16 años en el arte del escultismo. Los niños se
inscriben a un Club en su ciudad. En una ciudad puede haber varios
clubes de Conquistadores. Cada club tiene una directiva donde cada
miembro de esta es voluntario y tiene un rol pudiendo ser director,
subdirector, un secretario o tesorero. Cada club tiene su nombre, lema y
logo.

Los padres incriben a sus niños al Club de su preferencia por cercanía o
afinidad (por ejemplo en donde están sus amigos) en una clase según su
edad (no es inflexible). El niño ahora es miembro del club.

Una clase es un curso, estándar para todos los clubes, donde el niño
aprende diferentes cosas y debe realizar algunas actividades (llamados
Requisitos) obligatorias. Existen 7 clases reconocidas a nivel global;
cada clase tiene un nombre, color y logotipo. Cada niño pertenece a una
clase de la cual espera investirse. La investidura es la ceremonia donde
se entrega el reconocimiento a los niños que cumplieron todos los
requisitos de una clase. Las clases, aunque son progresivas, no son
requisito para tomar la siguiente.

Se necesita conocer los datos de los niños para cualquier situación que
pueda presentarse, entre otros: nombre, fecha de nacimiento, sexo,
padres y a quien llamar en caso de emergencia; también si padece de
alguna enfermedad (alergia, cardiopatía, etc.) pues las salidas al campo
y el ejercicio son frecuentes.

Es común que un niño al crecer y pasar tiempo en los clubes colabore
después en los clubes como instructor o en la directiva. También que al
formar una familia lleve a sus hijos a formar parte de los clubes.

Cada club tiene instructores para cada clase, de quien se desea conocer
también sus datos personales, cuales especialidades ha obtenido y de
cuales clases se ha investido. Esto para que cuando se necesite
voluntarios para un club se pueda encontrar mas fácilmente gente
capacitada.

Se desea saber la fecha de inicio y término de actividades de cada club.
Este periodo de trabajo puede variar en cada club, aunque en México,
normalmente inicia en octubre y termina en junio del año siguiente.
\emph{Por diversas circunstancias pueden empezar tan tarde como enero o
febrero. De la misma forma, aunque se espera que un club tenga al menos
20 reuniones en un año de trabajo, una a la semana, algunos clubes
avanzan mas rápido que otros, pudiendo terminar tan temprano como mayo o
tan tarde como octubre.}

Una reunión de Club puede durar entre 2 y 4 horas por lo que se realizan
varias actividades durante ella. \emph{Por lo pronto no es necesario
para el proyecto, pero se especifica por razones de contexto}.

Al realizar algunas actividades, los ``Conquistadores'' se reúnen con su
unidad. Cada unidad tiene un nombre, logotipo, un lema y, elegido de
entre los miembros, un capitán. Los niños de una unidad son del mismo
sexo. Cada unidad tiene un consejero que puede ser un aspirante a Guía
Mayor o un Guía Mayor investido y, como dormirá con su unidad en los
campamentos, también debe ser del mismo sexo que ésta.

El ``Conquistador'' puede realizar diferentes especialidades, donde cada
especialidad tiene una serie de requisitos con las cuales debe cumplir.
De cada especialidad tenemos su nombre, el año en que se aceptó en el
manual, el área a la que pertenece y un número secuencial dentro del
área a la que pertenece, con lo cual se forma su código. Se desea saber
la fecha en que se cumple con el requisito y con la especialidad. En
cada reunión, el consejero lleva el control de puntualidad y asistencia,
cuotas, tareas, uniforme y pulcritud de cada niño. Al final del
trimestre se desea conocer de cada clase, quien ha sido el mas cumplido.
Por las razones arriba mencionadas, de cada directivo, instructor y
consejero se desea conocer su historial de especialidades y de trabajo
en los clubes. Para estos cargos se requiere que la persona sea Guía
Mayor investido (haber cumplido con los requisitos la clase Guía Mayor).

\section{Ejemplos de información}\label{ejemplos-de-informaciuxf3n}

Ejemplos de información que se almacena: - Las 7 clases reconocidas son:
Amigo, Compañero, Explorador, Orientador, Viajero, Guía, Guía Mayor. -
En la ciudad de Culiacán existen entre otros los clubes: ``Águilas
Reales'' y ``Beta Crucis''. En Los Mochis: ``Delfines'' y ``Polaris'' -
El manual define al menos 6 áreas de especialidades: Habilidades
manuales (HM), Actividades Agrícolas (AA), Actividades Profesionales
(AP), Actividades Recreativas (AR), Estudio de la Naturaleza (EN) y
Habilidades Domésticas (AD).

\begin{itemize}
\tightlist
\item
  Ejemplos de especialidades definidas:
\end{itemize}

\begin{longtable}[]{@{}lllll@{}}
\toprule\noalign{}
Especialidad & Año & Área & Número & Nivel \\
\midrule\noalign{}
\endhead
\bottomrule\noalign{}
\endlastfoot
Arañas & 1928 & EN & 001 & 2 \\
Astronomía & 1928 & EN & 002 & 2 \\
Radioaficionado & 1928 & AP & 004 & 3 \\
Aves domésticas & 1928 & EN & 004 & 2 \\
Guitarra & 2012 & HR & 087 & 2 \\
Guitarra Avanzado & 2012 & HR & 088 & 3 \\
Fruticultura 1 & 1929 & AA & 008 & 2 \\
\end{longtable}

\begin{itemize}
\tightlist
\item
  El niño Juan Pérez del club Delfines de la clase ``Amigo'' aprendió a
  encender una fogata en forma segura. También cumplió con todos los
  requisitos de la especialidad de ``Avicultura'' y ``Pintura en
  vidrio''
\item
  La niña Teresa del club Águilas Reales cumplió con todos los
  requisitos de la especialidad ``Vida primitiva'' y ``Primeros Auxilios
  I''
\item
  Julio ha estado en los clubes ``Venados'' (2010-2013) de Guadalajara y
  ``Saraperos'' (2014-2018) de Saltillo, ahora lo han designado Director
  del club ``Mensajeros'' de Puebla.
\item
  En la Unidad ``Gorrioncillos'' del club Águilas Reales hay 12 niños,
  al menos uno de cada clase. El consejero es Renato que tiene 2 años
  colaborando en los clubes, aunque solo se invistió de las 3 últimas
  clases. Tiene un hijo en la clase de Amigo.
\item
  El club Águilas Reales tiene funcionando desde el año 1986. El club
  ``Hormigas'' tiene apenas 3 años funcionando, solo tiene niños en las
  clases de Amigo, Compañero y Viajero.
\item
  Esther ha pertenecido a los clubes desde 1990 y tiene 20 años
  colaborando en diferentes clubes, tiene 50 especialidades. Sus 5 hijos
  han pertenecido a los clubes y los mayores también han colaborado en
  los clubes. Este año ha sido instructora de ``Primeros Auxilios'',
  ``Conservación de alimentos'' y ``Vida primitiva'' para diferentes
  clubes.
\item
  Los requisitos de la especialidad de Avicultura, del área
  Agropecuaria, creada en 1928 con nivel 2 de dificultad son, entre
  otros:

  \begin{enumerate}
  \def\labelenumi{\arabic{enumi}.}
  \tightlist
  \item
    Tener la especialidad de ``Aves domésticas''
  \item
    Identificar por medio de imágenes, y conocer de forma general el uso
    de: a) Incubadoras d) Bebederos b) Perchas e) Comederos e) Aviarios
    f) Nidos
  \item
    ldentificar, en vivo o a partir de fotos, por lo menos dos razas de
    las siguientes especies, destacando su respectiva aptitud (carne,
    huevos, plumas y/o piel): a) Pato d) Codorniz b) Pollo e) Pavo e)
    Ganso f) Avestruz
  \item
    Describir un programa de alimentación para aves domésticas, desde la
    eclosión de los huevos hasta la etapa adulta.
  \end{enumerate}
\end{itemize}

\section{Consultas}\label{consultas}

Obtener la siguiente información (no están en orden de dificultad): 1.
Lista de asistencia del club Águilas Reales para el día 1 de noviembre
de 2020 2. Cargos (roles) que ha tenido Ramón Verdiales en los clubes
ordenado por año 3. Lista de clubes activos 4. Directores de los clubes
activos en el estado de Sinaloa ordenado por club 5. Lista de miembros
del club con progreso en cumplimiento de requisitos ordenado por clase y
por nombre resaltando a los que han cumplido al menos el 80\% de los
requisitos 6. Clubes de Jalisco con miembros que han cumplido al menos
el 80\% de los requisitos 7. Personas que han investido de Guía Mayor
que viven en Sinaloa 8. Personas que han obtenido la especialidad de
Aves Domésticas 9. Clubes con mas de 2 años de inactividad 10. Historial
de directores del club Saraperos de Saltillo 11. Año con más clubes
activos 12. Lista de miembros del club Polaris ordenado por cantidad de
especialidades obtenidas 13. Cantidad de clubes activos por estado 14.
Personas con mas de 10 años de servicio ininterrumpido en los clubes
ordenados por estado

\section{Reportes}\label{reportes-1}

\begin{enumerate}
\def\labelenumi{\arabic{enumi}.}
\tightlist
\item
  El director del club Polaris quiere obtener la lista de miembros
  activos que están sacando la especialidad de Avicultura y en que fecha
  cumplieron cada requisito
\item
  Manual de especialidades. Listado de áreas, listado de especialidades
  por área, requisitos de cada especialidad con información general de
  la especialidad de encabezado
\item
  El coordinador de clubes a nivel nacional quiere un histograma del
  tamaño de clubes activos
\item
  El director del club Saraperos quiere un reporte de asistencia de
  miembros del periodo actual agrupados por unidad con nombre del
  consejero
\item
  El instructor de la clase Amigo del club Delfines quiere saber los
  requisitos de clase cumplidos y no cumplidos por Juan Pérez. En el
  encabezado incluir el porcentaje de cumplimiento y el nombre del
  miembro, club e instructor. Precondición: Juan Pérez es miembro activo
  de esta clase y club.
\item
\end{enumerate}

\bookmarksetup{startatroot}

\chapter{``Morshu Estates''}\label{morshu-estates}

\begin{figure}[H]

{\centering \includegraphics{Inmobiliaria/Morshu_Estates.jpg}

}

\caption{Morshu Estates}

\end{figure}%

Hmmm\ldots{} ¡Bienvenido, amigo! ¡A ``Morshu Estates''! Aquí, si tienes
suficientes rupias, te consigo la mejor propiedad de toda la región.
Casas, castillos, cuevas misteriosas\ldots{} lo que sea, ¡todo está a la
venta o renta! Pero últimamente, manejar el negocio se ha vuelto\ldots{}
complicado.

\bookmarksetup{startatroot}

\chapter{¿Cómo funciona nuestro
negocio?}\label{cuxf3mo-funciona-nuestro-negocio}

\section{Prospección}\label{prospecciuxf3n}

Primero, está la prospección. ¡Tengo ojos en todas partes, amigo! Si hay
una propiedad disponible, Morshu lo sabe. Algunos dueños vienen a mí
rogando que venda su propiedad, otros\ldots{} bueno, digamos que
encuentro ``oportunidades de negocio''. Antes de poner algo en venta,
pruebo su calidad personalmenteee\ldots{} (si no explota o tiene
trampas, es buena).

\section{Publicación}\label{publicaciuxf3n}

Luego, viene la publicación. Antes, bastaba con gritar en la plaza:
``¡GRAN OPORTUNIDAD! ¡COMPRE AHORA!'' Pero la gente quiere más
detalles\ldots{} quieren fotos, descripciones, precios\ldots{}
¡demasiado trabajo! Ahora mando aprendices a pintar retratos de las
casas y escribir pergaminos detallados, pero eso lleva tiempo.

\section{Clientes}\label{clientes-1}

¡Y los clientes! ¡Ah, los clientes! Unos buscan una casita, otros
quieren una fortaleza\ldots{} pero todos quieren regatear. ¡No hay
regateo en ``Morshu Estates'', amigo! Pero claro, hay que llevar un
registro de quién busca qué\ldots{} y a veces me confundo. ¿Querías la
casa cerca del río o la guarida de los Moblins? ¡Necesito algo mejor que
anotarlo en pedazos de papel!

\section{Visitas}\label{visitas}

Después vienen las visitas. Aquí es donde todo se complica. A veces el
dueño no quiere abrir la puerta, a veces el cliente no aparece\ldots{}
¡y a veces la propiedad se cae antes de que lleguen! (Culpa del
constructor, no mía). Coordinar esto es un caos, ¡necesito un sistema
para organizar todo!

\section{Negociación}\label{negociaciuxf3n}

Cuando alguien decide comprar, viene la negociación. Aquí es donde
brillo, amigo. Se firman contratos, se pagan depósitos, a veces hay que
hacer ``ajustes'' a los papeles\ldots{} pero claro, si pierdo un
documento, problema para mí. Y sin un buen registro, a veces olvido
quién ya pagó y quién no.

\section{Tipos de Propiedades}\label{tipos-de-propiedades}

Ah, y no olvidemos que hay diferentes tipos de propiedades. ¡Un cliente
puede querer un castillo hoy y una choza en el bosque mañana! Necesito
organizar esto bien, porque ahora solo tengo una bolsa con pergaminos y
monedas mezcladas. Quiero saber saber su nombre, ubicación, superficie,
precio y el dueño actual. Y algunas cosas mas según el tipo de propiedad

\subsection{🏰 ``Fortalezas Morshu™'' (Castillos)
🏰}\label{fortalezas-morshu-castillos}

``¡Un castillo digno de un rey\ldots{} o de un villano con grandes
planes!''

\textbf{📜 Información importante:}

\begin{itemize}
\tightlist
\item
  Número de torres (Más torres = más prestigio\ldots{} y más lugares
  donde los intrusos pueden colarse).
\item
  Foso (¿Llena de agua? ¿Con cocodrilos? ¡Opciones premium!).
\item
  Trampas activas (¿Puentes trampa? ¿Paredes móviles? ¡Importante para
  la seguridad!).
\item
  Habitaciones secretas (No pregunten cómo las encontré\ldots{} pero
  están ahí).
\item
  Estado del trono (Algunos clientes solo compran si hay un trono
  decente).
\end{itemize}

\subsection{🏢 ``Cajas Altas Morshu™'' (Departamentos)
🏢}\label{cajas-altas-morshu-departamentos}

``¿Quieres vivir alto sin gastar tantas rupias? ¡Este es el lugar!''

\textbf{📜 Información importante:}

\begin{itemize}
\tightlist
\item
  Piso (¿El cliente quiere vista o prefiere no subir tantas escaleras?).
\item
  Acceso a elevador (*Si lo hay, ¡gran plus! Si no\ldots{} bueno, eso
  cuesta extra).
\item
  Número de vecinos molestos (¡Algunos edificios tienen más chismosos
  que habitantes!).
\item
  Reglas de la torre (¿Mascotas? ¿Fiestas? ¿Héroes con espadas? Todo
  debe estar claro).
\item
  Mantenimiento de los pasillos (¿Está limpio o huele a ruinas
  antiguas?).
\end{itemize}

\subsection{🏡 ``Casuchas Morshu™'' (Casas)
🏡}\label{casuchas-morshu-casas}

``¡Casitas acogedoras para héroes cansados y comerciantes con grandes
planes!''

\textbf{📜 Información importante:}

\begin{itemize}
\tightlist
\item
  Tamaño del terreno (Más grande = más espacio para rupias enterradas).
\item
  Estado del techo (¿Se filtra el agua? ¿Resiste ataques mágicos?).
\item
  Distancia al mercado más cercano (Porque nadie quiere caminar 5 horas
  para comprar pan).
\item
  Nivel de infestación de ratas (¡Algunas casas vienen con misiones
  secundarias incluidas!).
\item
  Sótano utilizable (Si el cliente quiere esconder cosas\ldots{} ¡digo,
  almacenar objetos!).
\end{itemize}

\subsection{🌴 ``Paraísos Morshu™'' (Villas)
🌴}\label{parauxedsos-morshu-villas}

``¡Para los que tienen MUCHAS rupias y quieren vivir como reyes!''

\textbf{📜 Información importante:}

\begin{itemize}
\tightlist
\item
  Número de piscinas (Una no es suficiente si tienes demasiadas rupias).
\item
  Distancia a la playa o lago (Si la vista no es buena, ¡descuenta
  rupias!).
\item
  Tipo de servidumbre disponible (Jardineros, cocineros, guardias
  personales\ldots{} ¡todo premium!).
\item
  Nivel de privacidad (¿Se puede ver desde la calle o es un escondite
  perfecto?).
\item
  Posibilidades de expansión (Algunos clientes quieren seguir
  construyendo\ldots{} o fortificándose).
\end{itemize}

\section{Posventa}\label{posventa}

Y por último\ldots{} la atención postventa. Algunos clientes vuelven a
comprar, otros necesitan mantenimiento\ldots{} y algunos desaparecen
misteriosamente (¡nada que ver conmigo!). Pero si los olvido, pueden
irse a otra inmobiliaria, ¡y Morshu NO PIERDE NEGOCIOS!

\bookmarksetup{startatroot}

\chapter{¿Qué necesito?}\label{quuxe9-necesito}

Hmmm\ldots{} ¡un sistema que haga esto más fácil! Algo que: - Registre
las propiedades sin que tenga que memorizar todo. - Guarde información
de dueños y clientes sin que use tres bolsas diferentes. - Me ayude a
coordinar visitas sin que haya confusión. - Mantenga un control claro de
las ventas y contratos. - Me dé reportes sobre quién ha pagado y quién
me debe.

¡Pero que sea fácil de usar, amigo! Si tengo que leer un manual\ldots{}
¡olvídalo! Ahora dime\ldots{} ¿puedes hacer un sistema así? Porque si
no\ldots{} no hay trato.

😏💰

\bookmarksetup{startatroot}

\chapter{Pawfect Care}\label{pawfect-care}

Empresa que ofrece el serivioc de Hotel de Cuidado de Mascotas con Venta
y Servicios Adicionales

\section{Descripción}\label{descripciuxf3n}

\subsection{Gestión de Mascotas}\label{gestiuxf3n-de-mascotas}

El sistema debe permitir registrar las mascotas que están bajo el
cuidado del hotel, incluyendo nombre, especie, raza, edad, historial
médico (vacunas, alergias), comportamiento y características especiales.
Además, debe registrar la mascota como huésped del hotel o como mascota
en venta, y actualizar el historial médico de cada mascota (vacunas,
tratamientos, enfermedades). También debe gestionar la disponibilidad de
mascotas en el hotel, ya sea para cuidado, venta o como mascotas de
compañía.

\subsection{Gestión de Clientes}\label{gestiuxf3n-de-clientes}

El sistema debe gestionar la información de los dueños de mascotas
(nombre, dirección, contacto) y asociarlos con sus mascotas. Además,
debe mantener un historial de reservas, compras y contratos de alquiler
de mascotas.

\subsection{Gestión de Reservas de
Cuidado}\label{gestiuxf3n-de-reservas-de-cuidado}

El sistema debe gestionar la reserva de espacios para el cuidado
temporal de mascotas, registrando fechas de entrada y salida, servicios
requeridos (alimentación, paseo, aseo) y costos asociados. También debe
ofrecer servicios adicionales como baños, entrenamiento y asistencia
veterinaria, que se puedan añadir a las reservas de cada mascota.

\subsection{Gestión de Ventas de Productos y
Mascotas}\label{gestiuxf3n-de-ventas-de-productos-y-mascotas}

El sistema debe gestionar un inventario de productos (alimentos,
juguetes, accesorios, medicinas) y mantener el stock actualizado.
Además, debe gestionar las transacciones de venta de productos,
registrando la información del cliente, productos comprados, cantidades
y pagos. También debe permitir la venta de mascotas disponibles en el
hotel, con información detallada sobre cada mascota (especie, raza,
edad, estado de salud).

\subsection{Servicio de Mascotas de
Compañía}\label{servicio-de-mascotas-de-compauxf1uxeda}

El sistema debe registrar las mascotas que están disponibles para el
servicio de compañía temporal, incluyendo estado de salud, especie,
raza, edad, comportamiento y disponibilidad. Además, debe permitir la
creación y gestión de contratos de alquiler para las mascotas de
compañía, incluyendo fechas de inicio y fin, cliente asociado, mascota
asignada, tarifas y estado del contrato (activo, completado, cancelado).
También debe registrar el historial de cada mascota de compañía, con
detalles de los clientes que la han alquilado y las fechas del servicio.

\subsection{Facturación y Pagos}\label{facturaciuxf3n-y-pagos}

El sistema debe generar facturas automáticamente para todas las
transacciones, ya sea de servicios de cuidado, ventas de productos,
ventas de mascotas o contratos de alquiler de mascotas de compañía.
Además, debe soportar múltiples tipos de pagos (efectivo, tarjeta de
crédito, transferencias electrónicas) y registrar un historial de todos
los pagos realizados por cada cliente.

\bookmarksetup{startatroot}

\chapter{MasterBall Tournaments}\label{masterball-tournaments}

MasterBall Tournaments es un sistema gestor de torneos de Pokemon Go. Se
necesita una base de datos para administrar la información de los
diferentes torneos a nivel mundial

\section{Registro y Gestión de
Equipos}\label{registro-y-gestiuxf3n-de-equipos}

\begin{itemize}
\tightlist
\item
  Formulario de inscripción en línea para equipos y jugadores.
\item
  Gestión de perfiles de equipos y jugadores con estadísticas y fotos.
  \#\# Calendario y Programación:
\item
  Generación automática de calendarios de partidos.
\item
  Notificaciones automáticas por correo electrónico o SMS para próximos
  partidos y cambios de horario. \#\# Resultados y Clasificaciones:
\item
  Actualización en tiempo real de resultados y tablas de clasificación.
\item
  Estadísticas detalladas de partidos y jugadores. \#\# Interacción y
  Comunicación:
\item
  Foro o chat para que los participantes y organizadores se comuniquen.
\item
  Publicación de noticias, fotos y videos de los partidos. \#\# Gestión
  de Sedes y Logística:
\item
  Geo-localización de sedes y asignación de canchas.
\item
  Gestión de recursos como árbitros y equipamiento. \#\# Monetización y
  Patrocinios:
\item
  Pasarela de pago para inscripciones y venta de entradas.
\item
  Espacios para publicidad y patrocinadores. \#\# Gamificación:
\item
  Recompensas y logros para equipos y jugadores destacados.
\item
  Sistema de puntos y rankings para fomentar la competencia. \#\#
  Seguridad y Privacidad:
\item
  Control de acceso y permisos para diferentes roles (administradores,
  árbitros, jugadores).
\item
  Protección de datos personales y cumplimiento de normativas de
  privacidad.
\end{itemize}

\bookmarksetup{startatroot}

\chapter{Salud y Vida}\label{salud-y-vida}

En \textbf{Salud y Vida} se necesita un sistema integral para
administrar toda la información relativa a los servicios que ofrece,
aunque inicialmente será solo una parte. Le han pedido diseñe una base
de datos que satisfaga sus necesidades de control de la información.

Para Salud y Vida, los clientes son importantes, son los que solicitan
servicios de la clínica para sí mismos o para sus familiares, por lo
cual se necesita conocer su nombre, apellidos, dirección y teléfono. Un
cliente puede tener asociados a varios dependientes.

Los pacientes pueden llegar a ventanilla y solicitar diferentes
servicios o agendar una cita por teléfono o por la web app.

El paciente llega a consulta con un médico general, el cual pregunta
sobre los síntomas actuales y en caso de ser necesario realiza un examen
físico. Si es la primera vez que asiste a Salud y Vida, le realiza
también una serie de preguntas sobre sus antecedentes médicos,
familiares y sociales, así como sobre cualquier tratamiento previo,
incluyendo enfermedades pasadas, cirugías, alergias y medicamentos que
esté tomando. En resumen hechos de salud del pasado y del problema
actual.

Si el médico puede diagnosticar con esta información puede indicar un
tratamiento que puede ser una serie de indicaciones o medicamentos, en
cuyo caso emite una receta. En algunos casos lo puede derivar a un
médico especialista de Salud y Vida o externo, en cuyo caso es dado de
alta.

Cualquiera de los médicos puede necesitar mas información para
disgnosticar, por lo que puede ordenar algún procedimiento extra:
análisis clinicos, biopsias o estudios de imagenología(rayos X,
tomografía, resonancia, etc).

La receta es una lista de medicamentos, dosis, frecuencia, duración e
indicaciones. Al imprimirse, lleva el nombre, firma y cédula del médico.

La receta se surte en la farmacia de Salud y Vida. El dependiente
verifica la receta, si es de Salud y Vida la coteja con el sistema, si
no simplemente la surte \emph{(Pendiente: medicamento controlado ¿Es
necesario hacer la distinción?)} y cobra.

Es necesario llevar un control del inventario de medicamentos.

\emph{Pendiente de redacción: presentación de medicamentos, dosis. Para
que el sistema le indique al dependiente con la dosis, frecuencia y
duración del tratamiento, cuantas cajas darle al paciente.}

Todos los servicios de Salud y Vida se proporcionan independientemente
si la orden es de un médico interno o externo. Se debe pagar antes de
recibir el servicio. Los servicios de análisis clínicos e imagenología
pueden ser consultados por los médicos o los clientes a través del
sistema. A los clientes se les entrega y envía por email un comprobante
con un número de servicio y una contraseña. Por email también se incluye
una URL para facilidad de acceso \emph{(Nota: esta URL no se guarda en
la base de datos. Esto es para otra materia)}.

Un paciente registrado mayor de edad puede consultar toda su información
a través de la web app. Un cliente puede hacerlo de sus dependientes
menores de edad.

Un cliente puede agregar a sus dependientes y todos los cargos derivados
de atenderlos serán a su cuenta, aunque el paciente puede pagar también.
\emph{(Mejorar redacción)}

*Pendiente: Hospitalización y cirujías**

\section{Análisis clínicos}\label{anuxe1lisis-cluxednicos}

Los resultados de los análisis llevan la siguiente información:

\textbf{Identificación del Laboratorio:} - Nombre y dirección del
laboratorio. - Información de contacto (teléfono, correo electrónico).

\textbf{Identificación del Paciente:} - Nombre completo. - Documento de
identidad. - Edad y sexo. - Número de historia clínica o expediente.

\textbf{Fecha y Hora:} - Fecha y hora de la toma de muestra. - Fecha y
hora de emisión del reporte.

\textbf{Detalles del Análisis:} - Nombre del análisis realizado. -
Método utilizado para el análisis. - Resultados obtenidos con sus
respectivas unidades de medida.

\textbf{Valores de Referencia:} - Rangos normales o valores de
referencia para cada análisis. - Comparación de los resultados del
paciente con los valores de referencia.

\textbf{Interpretación de Resultados:} - Comentarios o interpretaciones
del médico o especialista del laboratorio. - Observaciones adicionales
relevantes para la interpretación de los resultados.

\textbf{Responsable del Análisis:} - Nombre y firma del profesional que
realizó el análisis. - Número de cédula profesional.

\textbf{Observaciones Adicionales:} Cualquier información adicional que
pueda ser relevante para el diagnóstico o tratamiento del paciente.

Estos elementos aseguran que los resultados sean precisos, comprensibles
y útiles tanto para los pacientes como para los profesionales de la
salud

\section{Imagenología}\label{imagenologuxeda}

Los servicios de imagenología que se ofrecen son: - Rayos X - Tomografía
Computarizada (TC) - Resonancia Magnética (RM) - Ultrasonido (Ecografía)
- Mamografía - Densitometría Ósea

Un estudio de imagenología incluye varios elementos clave para asegurar
una evaluación precisa y completa del paciente.

\textbf{Identificación del Paciente:} - Nombre completo. - Documento de
identidad. - Edad y sexo. - Número de historia clínica o expediente.

\textbf{Detalles del Estudio:} - Tipo de estudio realizado (radiografía,
tomografía computarizada, resonancia magnética, ultrasonido, etc.). -
Fecha y hora del estudio. - Indicación clínica o motivo del estudio.

\textbf{Descripción Técnica:} - Parámetros técnicos utilizados (por
ejemplo, dosis de radiación, tipo de contraste utilizado). - Protocolo
de adquisición de imágenes.

\textbf{Imágenes Obtenidas:} - Imágenes en diferentes planos y cortes
según el tipo de estudio. - Anotaciones o marcas en las imágenes para
resaltar áreas de interés.

\textbf{Informe del Técnico:} - Descripción detallada de los hallazgos
observados en las imágenes. - Interpretación de los hallazgos en el
contexto clínico del paciente. - Diagnóstico diferencial si es
necesario. - Recomendaciones para estudios adicionales o seguimiento.

\textbf{Conclusión:} - Resumen de los hallazgos más importantes. -
Diagnóstico final si es posible.

\textbf{Firma y Credenciales:} - Nombre y firma del radiólogo
responsable del informe. - Número de cédula profesional.

\section{Servicios médicos}\label{servicios-muxe9dicos}

En Salud y Vida se ofrecen diferentes servicios médicos que aseguran que
los pacientes reciban una atención integral y especializada según sus
necesidades.

\textbf{Medicina General:} Atención primaria y seguimiento de pacientes
con diversas condiciones de salud.

\textbf{Pediatría:} Cuidado de la salud de los niños y adolescentes.

\textbf{Ginecología y Obstetricia:} Salud reproductiva de la mujer,
incluyendo embarazo, parto y enfermedades ginecológicas.

\textbf{Cardiología:} Diagnóstico y tratamiento de enfermedades del
corazón y el sistema circulatorio.

\textbf{Dermatología:} Tratamiento de enfermedades de la piel, cabello y
uñas.

\textbf{Endocrinología:} Manejo de trastornos hormonales y enfermedades
del sistema endocrino, como diabetes y problemas de tiroides.

\textbf{Gastroenterología:} Diagnóstico y tratamiento de enfermedades
del sistema digestivo.

\textbf{Ortopedia:} Tratamiento de lesiones y enfermedades del sistema
musculoesquelético.

\textbf{Neurología:} Diagnóstico y tratamiento de enfermedades del
sistema nervioso.

\textbf{Psiquiatría:} Tratamiento de trastornos mentales y emocionales.

\textbf{Oftalmología:} Cuidado de la salud visual y tratamiento de
enfermedades oculares.

\textbf{Urología:} Tratamiento de enfermedades del sistema urinario y,
en hombres, del sistema reproductor.

\section{Reportes}\label{reportes-2}

\begin{itemize}
\tightlist
\item
  Los médicos deben poder ver las citas que tienen programadas por día y
  por semana
\item
  Reporte de Inventario de medicamentos por categoría
\item
  Ingresos por especialidad por doctor
\item
\end{itemize}

\bookmarksetup{startatroot}

\chapter{TrumpetWare}\label{trumpetware}

\section{Descripción General}\label{descripciuxf3n-general}

TrumpetWare es un fabricante de productos de belleza y del hogar que
distribuye a través de embajadores. Debido al crecimiento reciente, es
necesario diseñar una Base de Datos para soportar una \textbf{Mobile
App} y una \textbf{Web App} que permitan gestionar las ventas, pedidos y
embajadores. Este caso práctico detalla los requisitos de la empresa.

\section{Requisitos del Sistema}\label{requisitos-del-sistema}

\subsection{Productos}\label{productos-1}

Cada producto tiene: - Nombre, clave, descripción. - Precio al público,
precio al embajador y costo de producción (no visible al embajador). -
Cada producto pertenece a una o más categorías: - Belleza, Limpieza del
hogar, Ropa, Jardinería, Organizadores. - Algunos productos se venden
solo en ciertas épocas del año y pueden estar descontinuados. - Durante
campañas, los productos pueden tener promociones (descuentos aplicados
sobre su precio de venta). \#\#\# Campañas y Temporadas - Las campañas
tienen una duración de 1 mes y no se traslapan. - Cada campaña incluye
un subconjunto de productos. - Tres campañas forman una temporada. - Las
temporadas ofrecen promociones adicionales a los embajadores que superen
niveles de ventas: - \$5000, \$10,000, \$20,000. \#\#\# Embajadores -
Los embajadores pueden ser de tres niveles: - Plata: Nivel inicial. -
Oro: Asesoran a embajadores Plata. - Platino: Asesoran a embajadores
Oro. - Para subir de nivel, un embajador debe mantener un nivel de
ventas por al menos 1 año. - Cada embajador: - Tiene una cartera de
clientes y pertenece a una región (con un CEDIS asignado). - Puede hacer
órdenes personales para adquirir productos a precio de embajador o
aprovechar promociones. - Establece metas personales de ventas. \#\#\#\#
Beneficios para embajadores: - Bonificaciones por desempeño: - Plata:
Acceso a productos en promoción por superar ventas de \$5000. - Oro:
Bono en efectivo por ventas mayores a \$10,000. - Platino: Acceso a
productos exclusivos por ventas superiores a \$30,000. - Recompensas no
monetarias: - Diplomas y reconocimientos digitales/físicos. - Cursos
gratuitos en marketing y liderazgo. - Invitaciones a eventos exclusivos.
- Concursos de temporada: - Los mejores embajadores pueden ganar viajes
o kits prémium. - Planes de antigüedad: - Descuentos adicionales y bonos
por trayectoria de 2, 5 y 10 años. \#\#\# Clientes De cada cliente se
guarda: - Nombre, teléfonos y direcciones geolocalizadas. - Un cliente
solo puede tener un embajador asignado. - Un cliente puede: - Pagar los
pedidos en abonos, siempre y cuando no tenga deudas pendientes mayores a
3 meses. - Hacer pedidos a su embajador o directamente en la página web.
- Los pagos se distribuyen de los pedidos más antiguos a los más nuevos.
\#\#\# Órdenes y Pedidos - Los embajadores agrupan los pedidos de sus
clientes en órdenes de compra: - Pueden realizar varias órdenes durante
una campaña. - En órdenes menores al mínimo establecido para su nivel,
se cobra un costo de envío. - Las órdenes pueden pagarse en abonos, pero
no se permite abonar si hay deudas previas. - Los clientes pueden
recibir los productos directamente o a través de su embajador. \#\#\#
Logística \#\#\#\# Regiones y CEDIS El territorio nacional está dividido
en regiones logísticas. - Cada región tiene un CEDIS (Centro de
Distribución), que gestiona: - Nombre, dirección y geolocalización. -
Recepción de productos desde la empresa. - Almacenamiento de inventario.
- Procesamiento y envío de órdenes. - Información del CEDIS: - Capacidad
máxima de almacenamiento. \#\#\#\# Procesos logísticos 1. Inventario: -
Cada CEDIS mantiene un inventario actualizado. - Productos en promoción
tienen prioridad de reabastecimiento. - Procesamiento de órdenes: -
Órdenes de embajadores: Enviadas al CEDIS de su región. - Órdenes de
clientes: Procesadas por el CEDIS correspondiente según la región del
cliente. 2. Envíos: - Plazos de entrega: 1-7 días hábiles. - Costos de
envío: - Gratuitos si las órdenes superan el mínimo según el nivel del
embajador. 3. Rastreo: - Los pedidos cuentan con un sistema de
seguimiento en tiempo real. 4. Optimización del sistema logístico -
Predicción de demanda por región para ajustar inventarios. -
Optimización de rutas para reducir costos de transporte. - Evaluación de
ventas por campaña para planificar reabastecimientos.

\subsection{Producción}\label{producciuxf3n}

Proceso de Producción La empresa produce sus propios artículos y
gestiona su fabricación con base en: Productos: La lista completa de
productos manufacturados. Recetas de producción: Cada producto tiene una
receta que especifica los materiales necesarios y su cantidad. Líneas de
producción: Fábricas o áreas designadas para fabricar productos
específicos. Tiempos de producción: El tiempo estimado en horas/días
para fabricar cada unidad del producto. Costo de producción: Calculado
con base en materiales, mano de obra y costos indirectos. Materiales Los
materiales necesarios para la producción incluyen: Nombre y descripción
del material. Unidad de medida (kg, litros, piezas, etc.). Costo por
unidad. Stock actual en el almacén de materias primas. Proveedor
asignado. Gestión de inventario de materias primas Se monitorea el
inventario de materiales en los almacenes. Si el inventario cae por
debajo del mínimo requerido, se genera una orden de compra al proveedor.
Los materiales son asignados a órdenes de producción según su
disponibilidad. Órdenes de Producción La producción se organiza en
órdenes de producción, que incluyen: Producto a fabricar. Cantidad
requerida. Fecha de inicio y finalización estimada. Línea de producción
asignada. Estado de la orden: Pendiente, En Proceso, Completada. Las
órdenes se generan con base en: La demanda proyectada para las campañas.
Inventario disponible en los CEDIS. Optimización de Producción
Pronóstico de demanda: Basado en ventas históricas y campañas futuras,
se ajusta el volumen de producción. Producción escalonada: Para
productos de alta demanda o con alta rotación, se producen lotes en
intervalos planificados. Capacidad de las líneas: Se considera la
capacidad diaria de las líneas de producción para evitar cuellos de
botella. Proveedores De cada proveedor se registra: Nombre, dirección,
datos de contacto. Materiales que suministra. Tiempos de entrega
promedio. Historial de cumplimiento de órdenes. Control de calidad Antes
de enviar los productos a los CEDIS, se realiza un control de calidad
que evalúa: Dimensiones, peso, y especificaciones del producto. Lotes
defectuosos se separan para ser reprocesados o desechados.

\subsection{Casos Prácticos}\label{casos-pruxe1cticos}

\subsubsection{Órdenes agrupadas:}\label{uxf3rdenes-agrupadas}

\begin{itemize}
\tightlist
\item
  Juan, un embajador nivel Plata, agrupa pedidos de tres clientes y
  realiza una orden de \$950. Se le aplica un costo de envío de \$100.
  Recibe los productos en 5 días.
\end{itemize}

\subsubsection{Promoción para
embajadores:}\label{promociuxf3n-para-embajadores}

\begin{itemize}
\tightlist
\item
  Rosa López, nivel Plata, alcanza \$12,000 en ventas y adquiere
  productos en promoción. Posteriormente, revende uno de ellos a precio
  normal.
\end{itemize}

\subsubsection{Adeudos de clientes:}\label{adeudos-de-clientes}

\begin{itemize}
\tightlist
\item
  Sandra Bullrock detecta que Javier Quezada no ha pagado un pedido de
  hace 3 meses. El cliente no puede realizar más compras hasta que
  regularice su deuda.
\end{itemize}

\subsubsection{Devolución:}\label{devoluciuxf3n}

\begin{itemize}
\tightlist
\item
  Cristina Van Rankin devuelve un producto defectuoso y recibe el
  reemplazo en 3 días.
\end{itemize}

\subsubsection{Demanda regional:}\label{demanda-regional}

\begin{itemize}
\tightlist
\item
  En la región norte, el CEDIS recibe reabastecimientos prioritarios
  para productos de limpieza debido a su alta demanda.
\end{itemize}

\subsubsection{Producción}\label{producciuxf3n-1}

\begin{enumerate}
\def\labelenumi{\arabic{enumi}.}
\tightlist
\item
  Falta de Materiales Críticos:
\end{enumerate}

\begin{itemize}
\tightlist
\item
  La línea de producción de ``Loción en Gel My Precious'' se detiene
  porque el inventario de aceite esencial de lavanda cae por debajo del
  mínimo requerido.
\item
  El sistema genera automáticamente una orden de compra al proveedor
  asignado, con un plazo de entrega estimado de 3 días.
\item
  Mientras tanto, el equipo de producción reorganiza la planificación
  para priorizar productos que no dependen de ese material.
\end{itemize}

\begin{enumerate}
\def\labelenumi{\arabic{enumi}.}
\setcounter{enumi}{1}
\tightlist
\item
  Producción Escalonada por Temporada:
\end{enumerate}

\begin{itemize}
\tightlist
\item
  Para la campaña ``Verano Fresco'', el sistema proyecta una alta
  demanda de la ``Colonia TW9 100ml''.
\item
  Se emiten 3 órdenes de producción escalonadas para fabricar 10,000
  unidades:

  \begin{itemize}
  \tightlist
  \item
    Primera orden: 4,000 unidades (completadas en mayo).
  \item
    Segunda orden: 4,000 unidades (junio).
  \item
    Última orden: 2,000 unidades (julio, para cubrir posibles pedidos
    tardíos).
  \end{itemize}
\end{itemize}

\begin{enumerate}
\def\labelenumi{\arabic{enumi}.}
\setcounter{enumi}{2}
\tightlist
\item
  Control de Calidad:

  \begin{itemize}
  \tightlist
  \item
    En una revisión de un lote de ``Baterías de Cocina Titanio'', el
    equipo de calidad detecta que 20 unidades presentan rayones.
  \item
    Las unidades defectuosas se separan y el sistema registra el lote
    como ``Parcialmente Aceptado'', mientras el resto se envía al CEDIS.
  \item
    El equipo de producción reprograma la fabricación de 20 unidades
    adicionales para reponer las dañadas.
  \end{itemize}
\item
  Optimización de Costos:
\end{enumerate}

Durante la campaña ``Amor en Tiempos del Covid'', el sistema sugiere
ajustar la receta de ``Talco Corporal Perfumado'' reemplazando un
material por otro de igual calidad, pero con menor costo. El cambio es
aprobado, reduciendo el costo de producción en un 5\% sin afectar las
ventas. Demanda no prevista:

Un artículo nuevo, ``Organizador Multifuncional'', tiene una demanda
30\% mayor a lo proyectado en su primera campaña. El sistema detecta el
agotamiento del inventario en los CEDIS y genera una orden de producción
urgente. El área de producción reorganiza las líneas y fabrica 5,000
unidades adicionales en 2 semanas. Pronóstico Erróneo:

Un exceso de inventario de ``Velas Aromáticas Pure Serenity'' tras la
campaña ``Otoño Místico'' obliga a la empresa a realizar ajustes. Las
velas restantes se redistribuyen a las siguientes campañas como
productos en promoción, lo que ayuda a minimizar pérdidas. Pedidos
Especiales:

Un gran cliente solicita 1,000 unidades personalizadas del ``Eau
d'parfum''. Se genera una orden de producción específica con su propio
número de lote. La producción se prioriza, y el cliente recibe el pedido
en tiempo y forma.




\end{document}
